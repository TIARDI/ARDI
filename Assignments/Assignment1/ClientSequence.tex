%!TEX root = Main.tex
\documentclass[Main]{subfiles}

\begin{document}

\subsection{Sequence diagrams}

The work of the client is mostly performed by the eventhandlers, each creating a connection to the server during initialization. After initialization, the the client performs one iteration of eventhandling, that is a call to \code{Reactor::handle\_events()} which was explained in figure \ref{seq:reactorSeq}. Thus, the sockets created by each of the eventhandlers will be writeable and each of the eventhandlers will hande the \code{WRITE} event. The sequence for the eventhandlers is outlined in figure \ref{fig:clientEventHandler}.

\begin{figure}
\begin {sequencediagram}
	\newinst[0]{rea}{:Reactor}
	\newinst[0.4]{evh}{:EventHandler}
	\newinst[1.1]{sock}{:SOCK\_stream}

	\begin{messcall}{rea}{Handle\_event(h, eType)}{evh}

		\begin{sdblock}{if}{eType == WRITE}
			\begin{call}{evh}{send(\dots)}{sock}{res}
			\end{call}

		\end{sdblock}

		\begin{sdblock}{else}{}
			\begin{call}{evh}{remove\_handler(this, eType)}{rea}{}
			\end{call}
		\end{sdblock}

	\end{messcall}



\end{sequencediagram}

\caption{Sequence for client event handling.}
\label{fig:clientEventHandler}
\end{figure}


\end{document}