%!TEX root = Main.tex
\documentclass[Main]{subfiles}

\begin{document}

\begin{figure}
\begin {sequencediagram}
	\newthread {main}{: Main }
	\newinst[1.6]{rea}{: Reactor}
	\newinst[1.8]{demu}{: DemuxTable}
	\newinst[0.4]{wins}{:WinSock2}
	\newhiddenthread{hid}{:hidden}

	\begin{sdblock}{Run Loop}{Main loop}
		\begin{messcall}{main}{handle\_events()}{rea}
			\begin{messcall}{rea}{convert\_to\_fd\_sets()}{demu}
			\end{messcall}

			\begin{call}{rea}{select(\dots)}{wins}{ret}
			\end{call}

			\begin{sdblock}{if}{ret == SOCKET\_ERROR}
				\begin{callself}{rea}{throw exception}{}
				\end{callself}
			\end{sdblock}

			\begin{sdblock}{if}{ret == 0}
				\begin{callself}{rea}{Handle Timeout}{}
				\end{callself}
			\end{sdblock}

			\begin{sdblock}{if}{else}
				\begin{messcall}{rea}{handle\_event(first, type) - depending on type}{hid}
				\end{messcall}
			\end{sdblock}

		\end{messcall}
	\end{sdblock}

\end{sequencediagram}

\caption{Sequence diagram for Server}
\label{seq:serverMain}
\end{figure}

The \code{handle\_event(first, type)} will call the class of the given type.
This means it can call the one of the following three classes:
\begin{itemize}
	\item \code{Alarm\_EventHandler}
	\item \code{Log\_EventHandler}
	\item \code{PatientValue\_EventHandler}
\end{itemize}

\newpage
\begin{figure}
\begin {sequencediagram}
	\newhiddenthread {hid}{:hidden}
	\newinst[0.4]{alm}{:Alarm\_EventHandler}
	\newinst[0.2]{sock}{:SOCK\_stream}
	\newthreadmargin[3.5]{rea}{:Reactor}

	\begin{messcall}{hid}{Handle\_event(h, eType)}{alm}

		\begin{sdblock}{if}{eType == READ}

			\begin{messcall}{alm}{remove\_handler(this, eType)}{rea}
			\end{messcall}
		\end{sdblock}

		\begin{sdblock}{else}{}
			\begin{messcall}{alm}{remove\_handler(this, eType)}{rea}
			\end{messcall}
		\end{sdblock}

	\end{messcall}



\end{sequencediagram}

\caption{Title of figure}
\label{fig:alarmEventHandler}
\end{figure}



\end{document}