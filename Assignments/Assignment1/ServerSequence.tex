%!TEX root = Main.tex
\documentclass[Main]{subfiles}

\begin{document}

\begin{figure}
\begin {sequencediagram}
	\newhiddenthread {hid}{:hidden}
	\newinst[0.4]{alm}{:Alarm\_EventHandler}
	\newinst[1]{rea}{:Reactor}

	\begin{call}{hid}{Reactor}{rea}{}
	\end{call}
	\begin{messcall}{hid}{Handle\_event(h, eType)}{alm}
	\end{messcall}

		% 	\begin{sdblock}{if}{eType == READ}
		% 	\message{alm}{remove\_handler(this, eType)}{rea}
		% \end{sdblock}

		% \begin{sdblock}{else}{}
		% 	\message{alm}{remove\_handler(this, eType)}{rea}
		% \end{sdblock}

\end{sequencediagram}

\caption{Title of figure}
\label{fig:alarmEventHandler}
\end{figure}



The \code{handle\_event(first, type)} will call the class of the given type.
This means it can call the one of the following three classes:
\begin{itemize}
	\item \code{Alarm\_EventHandler}
	\item \code{Log\_EventHandler}
	\item \code{PatientValue\_EventHandler}
\end{itemize}

\newpage


\end{document}