%!TEX root = Main.tex
\documentclass[Main]{subfiles}

\begin{document}

\subsection{Implementation details}
\subsubsection{Winsock Handleing}
When working with Winsock there is a need to initialize the Winsock library.
This is done by calling the \code{WSAStartup()} function, and calling \code{WSASClose()} when sockets are no longer needed.\\
To ensure the sockets are initialized and properly cleaned up afterwards, 
we have implemented some helper functions \code{winsockHandling::init\_winsock()} and \code{winsockHandling::close\_winsock()}.\\
When WinSock are initialized a mutex is taken to prevent race conditions.
If it is the first time the function is called, Winsock is initialized, and a reference counter is incremented.\\
When cleaning the mutex is once again taken, and reference counter is decremented. When the counter reaches 0, \code{WSASClose} is called.

\end{document}