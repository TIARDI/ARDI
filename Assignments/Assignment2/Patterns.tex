%!TEX root = Main.tex
\documentclass[Main]{subfiles}

\begin{document}

\section{Patterns used in the solution}\label{sec:patterns}
This section describes the different patterns used in this solution, and what they are used for. 
It should be noted that through exercise one to three, the patterns WrapperFaçade, Reactor, Bridge and Singleton where used, and so won't be mentioned further, unless they are required to. We choose to only specify the new patterns used.

\subsection{Acceptor/Connector}
The acceptor/connector pattern were already used to a certain degree in exercise three, however that was only a pseudo pattern, not the POSA2 pattern. 
Here the full pattern was implemented.
The Acceptor handles the server-side dispatching, accessing the service that should be handled, based on what event is raised.
The Connector registers the services on the client side, so that when a new connection should occur, the dispatcher would access the service handler to figure out, what to do with that type of connection.

\subsection*{Leader/Follower}
The Leader/Follower pattern handles the multi-threading of the server. The concept deals with a thread queue, and three different states; Leader, Follower and Processing. In the thread pool, a single thread is the leader, and the one that will handle incoming events. When an incoming event comes along, the leader thread takes it, elects a new leader and processes the event. If another event comes along, while the old leader processes the previous event, the new leader handles that event and elects a new leader and so on. When a processing thread is done, it checks to see if anyone is leader. If there is, then it becomes a follower, otherwise it becomes a leader.

\end{document}
