%!TEX root = Main.tex
\documentclass[Main]{subfiles}

\begin{document}

\section{Discussion of architecture decisions}

To create a server and a client, we implement them using the Acceptor/connector pattern along with the Reactor pattern.

With this configuration all the connection will be as easy as possible for the clients.
By using the TCP/IP-protocol, the client do not need anything special to establish a connection since this is standard for all platforms.
This also means that the server must apply to the TCP/IP-protocol and other clients will be able to communicate with it through a given port.


The server will be using different patterns from the TIARDI lessons:
\begin{itemize}
	\item Acceptor/Connector
	\item Wrapper Facade
	\item Reactor
	\item \fxnote{Half-(a)Sync or Leader/follow}
\end{itemize}

The Wrapper Facade pattern from Exercise 1 is still used independently.
The Reactor, as well from Exercise 1, can be used separately with the wrapper has been applied to the functions. 
The Acceptor/Connector is very closely linked to the Reactor and will be very difficult to use without it.
\fxnote{Something about the sync or leader}


Building Exercise 4 was done in Windows, since it only was build in Microsoft's Visual Studio.	



\end{document}