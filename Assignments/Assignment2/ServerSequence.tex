%!TEX root = Main.tex
\documentclass[Main]{subfiles}

\begin{document}

\subsection{Sequence diagram}
Figure \ref{seq:reactorSeq} show the \code{Reactor}'s sequence diagram.
The \code{handle\_event(HANDLE, type)} will call the class of the given type.
This means it can call the one of the following three classes:
\begin{itemize}
	\item \code{Alarm\_EventHandler}
	\item \code{Log\_EventHandler}
	\item \code{PatientValue\_EventHandler}
\end{itemize}

These calls are all shown in Figure \ref{fig:SererEventHandler} as a common sequence diagram.
\newpage
\begin{figure}[htbp]
\begin {sequencediagram}
	\newhiddenthread {hid}{:hidden}
	\newinst[0.4]{alm}{:EventHandler}
	\newinst[0.5]{sock}{:SOCK\_Acceptor}
	\newinst{pait}{:Patient\_Handler}

	% \newthreadmargin[1.7]{rea}{:Reactor}

	\begin{messcall}{hid}{Handle\_event(h, eType)}{alm}

		\begin{callself}{alm}{Accept()}{}
			\begin{callself}{alm}{make\_service\_handler}{SERVICE\_HANDLER}
			\end{callself}

			\begin{callself}{alm}{accept\_service\_handler}{}
				\begin{call}{alm}{accept(handler->peer())}{sock}{}
				\end{call}
			\end{callself}

			\begin{callself}{alm}{activate\_service\_handler(\dots)}{}
				\begin{call}{alm}{open()}{pait}{}
				\end{call}
			\end{callself}
		\end{callself}

		% \begin{sdblock}{if}{eType == READ}
		% 	\begin{call}{alm}{recv(\dots)}{sock}{res}
		% 	\end{call}

		% 	\begin{sdblock}{if}{res == 0}
		% 		\begin{messcall}{alm}{remove\_handler(this,eType)}{rea}
		% 		\end{messcall}

		% 		\mess{alm}{return}{hid}

		% 	\end{sdblock}

		% 	\begin{callself}{alm}{PrintMessage}{}
		% 	\end{callself}
		% \end{sdblock}

		% \begin{sdblock}{else}{}
		% 	\begin{call}{alm}{remove\_handler(this, eType)}{rea}{}
		% 	\end{call}
		% \end{sdblock}

	\end{messcall}

\end{sequencediagram}

\caption{Server's EventHandler}
\label{fig:SererEventHandler}
\end{figure}




\end{document}