%!TEX root = Main.tex
\documentclass[Main]{subfiles}

\begin{document}

\subsection{Intro to design of client}

The client is designed to be a single program, that sends messages corresponding to the three event types, that the server must support. These are described in the assignment as such:
\textit{
\begin{itemize}
\item an alarm event (carrying a priority field and a text string).
\item a patient value event (carrying a type field and a value field).
\item a log event (carrying a text string).
\end{itemize}}

The design used for this exercise uses a connection for each of the event types. Thus the client contains three eventhandlers - one for each type of event. A reactor is used to demultiplex the "write" events of the associated sockets. Each handler uses the wrapper facade classes \code{Sock\_Connector} and \code{Sock\_stream} to handle connections to the server.

\end{document}